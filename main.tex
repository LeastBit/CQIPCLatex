% !TEX encoding = UTF-8
\documentclass[a4paper]{article}

%%%%%%%%%%%%%%%%%所用宏包%%%%%%%%%%%%%%%%%%%%
\usepackage{ctex}
\usepackage{fontspec}
\usepackage{graphicx}
\usepackage{ulem}
\usepackage[top=3cm,bottom=2.5cm,left=3cm,right=3cm]{geometry}
\usepackage{fancyhdr}  % 页眉页脚
\usepackage{titlesec}  % 标题格式
\usepackage{titletoc}  % 目录格式
\usepackage{caption}   % 图表标题
\usepackage{enumitem}  % 列表环境
\usepackage{tocloft}   % 目录格式
\usepackage{booktabs}  % 表格美化
\usepackage{natbib}    % 参考文献格式化
\usepackage{gbt7714}   % 添加 GB/T 7714-2015 参考文献样式支持
\usepackage{hyperref}  % 超链接
\usepackage{setspace}  % 用于设置行距
\usepackage{amsmath} % 加载 amsmath 宏包以支持更复杂的数学公式

%%%%%%%%%%%%%%%%%样式设置%%%%%%%%%%%%%%%%%%%%
% hyperref 设置
\hypersetup{
    colorlinks=true,       % 彩色链接
    linkcolor=black,       % 目录链接颜色
    citecolor=blue,        % 引用链接颜色
    urlcolor=blue,         % URL链接颜色
    bookmarksopen=true,    % 展开书签
    bookmarksnumbered=true % 书签带编号
}

% 页码设置 - 使用大写罗马数字
\renewcommand{\thepage}{\Roman{page}}  % 大写罗马数字页码

% 设置字体
\setmainfont{Times New Roman}
% 自定义字体路径(使用不同的族名避免冲突)
\setCJKfamilyfont{mysong}{simsun.ttc}
\setCJKfamilyfont{myhei}{simhei.ttf}
\setCJKfamilyfont{myli}{simli.ttf}
\newcommand{\mysong}{\CJKfamily{mysong}}
\newcommand{\myhei}{\CJKfamily{myhei}}
\newcommand{\myli}{\CJKfamily{myli}}

% 优化下划线对齐
\newcommand{\infounderline}[2][8cm]{\uline{\makebox[#1][c]{#2}}}

% 修改目录标题,不使用自动的"目录"标题
\renewcommand{\contentsname}{}

% === 目录格式设置 ===
% 设置目录点间距
\renewcommand{\cftdot}{.}
\renewcommand{\cftdotsep}{1}  % 点之间的距离

% 移除目录间距设置,改为使用固定行距
\setlength{\cftbeforesecskip}{0pt}  % 移除间距
\setlength{\cftbeforesubsecskip}{0pt}  % 移除间距
\setlength{\cftbeforesubsubsecskip}{0pt}  % 移除间距

% 设置目录格式 - section
\renewcommand{\cftsecfont}{\mysong\zihao{-4}}  % 一级条目字体
\renewcommand{\cftsecleader}{\cftdotfill{\cftdotsep}}  % 一级条目引导符
\renewcommand{\cftsecpagefont}{\mysong\zihao{-4}}  % 一级条目页码字体

% 设置目录格式 - subsection
\renewcommand{\cftsubsecfont}{\mysong\zihao{-4}}  % 二级条目字体
\renewcommand{\cftsubsecleader}{\cftdotfill{\cftdotsep}}  % 二级条目引导符
\renewcommand{\cftsubsecpagefont}{\mysong\zihao{-4}}  % 二级条目页码字体

% 设置目录格式 - subsubsection
\renewcommand{\cftsubsubsecfont}{\mysong\zihao{-4}}  % 三级条目字体
\renewcommand{\cftsubsubsecleader}{\cftdotfill{\cftdotsep}}  % 三级条目引导符
\renewcommand{\cftsubsubsecpagefont}{\mysong\zihao{-4}}  % 三级条目页码字体

% 设置缩进 - 根据要求缩进
\setlength{\cftsecindent}{0em}
\setlength{\cftsubsecindent}{2em}  % 改为2
\setlength{\cftsubsubsecindent}{4em}  % 改为4

% 一级无序标题(摘要、参考文献、致谢等)- 添加目录条目
\newcommand{\chapternonumberformat}[2]{
  \phantomsection\vspace*{12pt}
  \begin{center}
    \myhei\zihao{3}#1
  \end{center}
  \vspace*{12pt}
  % 添加目录条目但不显示编号
  \addcontentsline{toc}{section}{#1}\label{#2}  % 添加标签以便引用
}

% 一级标题(带编号)
\titleformat{\section}
{\myhei\zihao{3}\centering}
{\thesection}{1em}{}
\titlespacing*{\section}{0pt}{24pt}{18pt}

% 二级标题
\titleformat{\subsection}
{\myhei\zihao{-3}}
{\thesubsection}{1em}{}
\titlespacing*{\subsection}{0pt}{20pt}{16pt}

% 三级标题
\titleformat{\subsubsection}
{\myhei\zihao{4}}
{\thesubsubsection}{1em}{}
\titlespacing*{\subsubsection}{0pt}{16pt}{12pt}

% 图表标题格式
\DeclareCaptionFont{mysong}{\mysong\zihao{5}}
\captionsetup[figure]{font=mysong, position=below, skip=5pt, name=图}
\captionsetup[table]{font=mysong, position=above, skip=5pt, name=表}

% 正文格式
\newcommand{\maintext}{\mysong\zihao{-4}}
\newcommand{\referencetext}{\mysong\zihao{5}}

% 设置首行缩进
\setlength{\parindent}{2em}
\setlength{\parskip}{0pt}  % 段落间距

% 设置行距
\linespread{1.25}  % 大约20磅行距

% 重新定义enumerate和itemize环境
\setlist{noitemsep}

%%%%%%%%%%%%%%%%%%%文档%%%%%%%%%%%%%%%%%%%%%%
\begin{document}
%%%%%%%%%%%%%%%%%%%封面%%%%%%%%%%%%%%%%%%%%%%
\thispagestyle{empty}
% 精确垂直间距控制
\vspace*{2\baselineskip}
\begin{center}
  {\myli\fontsize{27}{27}\selectfont 重庆工业职业技术学院}
  
  \vspace*{5\baselineskip}

  {\myhei\fontsize{36}{36}\selectfont 毕业论文}
  
  \vspace*{3\baselineskip}
  
  \includegraphics[width=5cm,height=5cm]{images/Logo.png}
  
  \vspace*{3\baselineskip}
  
  {\mysong\zihao{4}
  \begin{tabular}{@{}rl@{}}
      课题名称:&\begin{tabular}{@{}c@{}}
                  \infounderline{网络表示学习中超参数分配的}\\
                  \infounderline{多级树映射}
                \end{tabular} \\[0.5cm]
      专业班级:&\infounderline{23计网(专本)302} \\[0.5cm]
      学生姓名:&\infounderline{李豪} \\[0.5cm]
      指导老师:&\infounderline{傅顺}
  \end{tabular}}
  
  \vspace*{6\baselineskip}

  {\mysong\zihao{4}二〇二五年六月}
\end{center}

%%%%%%%%%%%%%%%%%%%摘要%%%%%%%%%%%%%%%%%%%%%%
\newpage
\setcounter{page}{1}  % 页码重新开始
\chapternonumberformat{摘要}{chap:abstract}
\maintext%
摘要内容\ldots{}(这里填写论文摘要)

% \vspace{1cm}
\noindent\textbf{关键词:}关键词1;关键词2;关键词3;关键词4

% \newpage
% \chapternonumberformat{Abstract}{chap:abstract-en}
% \maintext
% Abstract content\ldots{}(英文摘要内容)
% \vspace{1cm}
% \noindent\textbf{Keywords:} keyword1; keyword2; keyword3; keyword4

%%%%%%%%%%%%%%%%%%%目录%%%%%%%%%%%%%%%%%%%%%%
\newpage
\phantomsection% 为超链接创建锚点
\begin{center}%
  \myhei\zihao{3}目录
\end{center}%
\vspace*{12pt}
\addcontentsline{toc}{section}{目录}\label{chap:toc}

% 设置目录行距为固定20磅
\begingroup
% 完全固定行距为20磅
\renewcommand{\baselinestretch}{1.0}
\setlength{\baselineskip}{20pt}
\setlength{\parskip}{0pt}
\tableofcontents
\endgroup

% 从这里开始使用阿拉伯数字页码
\newpage
\pagenumbering{arabic}  % 切换为阿拉伯数字页码
\setcounter{page}{1}    % 页码重新从1开始

%%%%%%%%%%%%%%%%%%%正文%%%%%%%%%%%%%%%%%%%%%%
\section{绪论}
\subsection{国内研究背景}
\maintext%
国内研究背景内容\cite{fu_cost-sensitive_2024}\cite{hao_multi-objective_2024}\ldots{}

\subsubsection{结论1}
\maintext%
结论1内容\ldots{}
\begin{equation}
  \alpha = \min{(a - b)}
  \label{eq:alpha}
\end{equation}

% 图表示例
\begin{figure}[htbp]%
  \centering
  \includegraphics[width=0.7\textwidth]{images/logo.png}
  \caption{图示例说明}\label{fig:example}
\end{figure}

\begin{table}[htbp]%
  \centering
  \caption{表格示例}\label{tab:example}
  \begin{tabular}{ccc}
    \toprule
    列1 & 列2 & 列3 \\
    \midrule
    数据1 & 数据2 & 数据3 \\
    数据4 & 数据5 & 数据6 \\
    \bottomrule
  \end{tabular}
\end{table}



%%%%%%%%%%%%%%%%%%%%参考文献%%%%%%%%%%%%%%%%
\newpage
\renewcommand{\thepage}{\arabic{page}}%
% 设置参考文献格式
\bibliographystyle{gbt7714-numerical}% 使用 GB/T 7714-2015 格式
{\referencetext% 使用定义的参考文献字体和字号
\bibliography{mybib}% 参考文献数据库文件
}

%%%%%%%%%%%%%%%%%%%%致谢%%%%%%%%%%%%%%%%%%%%%
\newpage
\chapternonumberformat{致谢}{chap:acknowledgment}
\maintext%
致谢内容\ldots{}

%%%%%%%%%%%%%%%%%%%%附录%%%%%%%%%%%%%%%%%%%%%
\newpage
\chapternonumberformat{附录}{chap:appendix}
\maintext%
附录内容\ldots{}
\end{document}